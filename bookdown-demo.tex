% Options for packages loaded elsewhere
\PassOptionsToPackage{unicode}{hyperref}
\PassOptionsToPackage{hyphens}{url}
%
\documentclass[
]{book}
\usepackage{lmodern}
\usepackage{amssymb,amsmath}
\usepackage{ifxetex,ifluatex}
\ifnum 0\ifxetex 1\fi\ifluatex 1\fi=0 % if pdftex
  \usepackage[T1]{fontenc}
  \usepackage[utf8]{inputenc}
  \usepackage{textcomp} % provide euro and other symbols
\else % if luatex or xetex
  \usepackage{unicode-math}
  \defaultfontfeatures{Scale=MatchLowercase}
  \defaultfontfeatures[\rmfamily]{Ligatures=TeX,Scale=1}
\fi
% Use upquote if available, for straight quotes in verbatim environments
\IfFileExists{upquote.sty}{\usepackage{upquote}}{}
\IfFileExists{microtype.sty}{% use microtype if available
  \usepackage[]{microtype}
  \UseMicrotypeSet[protrusion]{basicmath} % disable protrusion for tt fonts
}{}
\makeatletter
\@ifundefined{KOMAClassName}{% if non-KOMA class
  \IfFileExists{parskip.sty}{%
    \usepackage{parskip}
  }{% else
    \setlength{\parindent}{0pt}
    \setlength{\parskip}{6pt plus 2pt minus 1pt}}
}{% if KOMA class
  \KOMAoptions{parskip=half}}
\makeatother
\usepackage{xcolor}
\IfFileExists{xurl.sty}{\usepackage{xurl}}{} % add URL line breaks if available
\IfFileExists{bookmark.sty}{\usepackage{bookmark}}{\usepackage{hyperref}}
\hypersetup{
  pdftitle={IDA Introduktion til R},
  pdfauthor={Tue Hellstern},
  hidelinks,
  pdfcreator={LaTeX via pandoc}}
\urlstyle{same} % disable monospaced font for URLs
\usepackage{longtable,booktabs}
% Correct order of tables after \paragraph or \subparagraph
\usepackage{etoolbox}
\makeatletter
\patchcmd\longtable{\par}{\if@noskipsec\mbox{}\fi\par}{}{}
\makeatother
% Allow footnotes in longtable head/foot
\IfFileExists{footnotehyper.sty}{\usepackage{footnotehyper}}{\usepackage{footnote}}
\makesavenoteenv{longtable}
\usepackage{graphicx,grffile}
\makeatletter
\def\maxwidth{\ifdim\Gin@nat@width>\linewidth\linewidth\else\Gin@nat@width\fi}
\def\maxheight{\ifdim\Gin@nat@height>\textheight\textheight\else\Gin@nat@height\fi}
\makeatother
% Scale images if necessary, so that they will not overflow the page
% margins by default, and it is still possible to overwrite the defaults
% using explicit options in \includegraphics[width, height, ...]{}
\setkeys{Gin}{width=\maxwidth,height=\maxheight,keepaspectratio}
% Set default figure placement to htbp
\makeatletter
\def\fps@figure{htbp}
\makeatother
\setlength{\emergencystretch}{3em} % prevent overfull lines
\providecommand{\tightlist}{%
  \setlength{\itemsep}{0pt}\setlength{\parskip}{0pt}}
\setcounter{secnumdepth}{5}
\usepackage{booktabs}
\usepackage{amsthm}
\makeatletter
\def\thm@space@setup{%
  \thm@preskip=8pt plus 2pt minus 4pt
  \thm@postskip=\thm@preskip
}
\makeatother
\usepackage[]{natbib}
\bibliographystyle{apalike}

\title{IDA Introduktion til R}
\author{Tue Hellstern}
\date{2020-04-18}

\begin{document}
\maketitle

{
\setcounter{tocdepth}{1}
\tableofcontents
}
\hypertarget{webinar}{%
\chapter{Webinar}\label{webinar}}

Webinar den 22-04-2020

Kl. 19:00

Tue Hellstern

Jelgreen Consult

\hypertarget{intro}{%
\chapter{Introduktion}\label{intro}}

\hypertarget{R_RStudio}{%
\chapter{R og RStudio}\label{R_RStudio}}

Du skal have installeret R og det IDE udviklingsmiljø, der hedder RStudio, begge er open source og begge findes til Mac, Linux og Windows.

\hypertarget{r}{%
\section{R}\label{r}}

Du skal downloade R fra
https://cran.r-project.org
Du skal her vælge den version, der passer til din computer -- Linux, OS X (Mac) eller Windows.
Der er vejledninger for de forskellige installationer på www siden.

\hypertarget{rstudio}{%
\section{RStudio}\label{rstudio}}

Når R er installeret, kan du bruge det, men for at gøre det nemmere bruger vi et udviklingsmiljø, der hedder RStudio. Rstudio findes i flere forskellige versioner, den du skal bruge er \emph{RStudio Desktop Open Source License}

Som du finder her: https://www.rstudio.com/products/rstudio/download/
Du skal her vælge den version, der passer til din computer.

\includegraphics{img/rstudio_editor.png}

\hypertarget{grund}{%
\chapter{Grundlæggende R}\label{grund}}

\hypertarget{data}{%
\chapter{Data}\label{data}}

\hypertarget{plots}{%
\chapter{Plots}\label{plots}}

\hypertarget{shiny}{%
\chapter{Demo Shiny}\label{shiny}}

Når du har lavet en ''analyse'' i R vil et tit være en fordel at kunne dele denne med andre, det kan du gøre ved at sende den dit R script og data, men det er nok ikke den mest brugervenlige måde.
En anden ulighed er vise det på en webside så det kan tilgås via en browser. Det kan du opnå via en R pakke der hedder shiny.
Du kan finde en del eksempler på løsninger oprettet i shiny her: shiny.rstudio.com/gallery/\#demos

Det er også muligt at afvikle de 11 eksempler der følger med shiny på følgende måde:

\begin{verbatim}
# Shiny eksempler
runExample("01_hello")
runExample("02_text")
runExample("03_reactivity")
runExample("04_mpg")
runExample("05_sliders")
runExample("06_tabsets")
runExample("07_widgets")
runExample("08_html")
runExample("09_upload")
runExample("10_download")
runExample("11_timer")
\end{verbatim}

\hypertarget{struktur-af-en-shiny-app}{%
\section{Struktur af en Shiny app}\label{struktur-af-en-shiny-app}}

En Shimy App er ''samlet'' i et enkelt script app.R som du kan afvikle.
app.R består af tre dele:

\begin{itemize}
\tightlist
\item
  Userinterface (ui.R) - \emph{Her styrer du layout af din applikation}
\item
  Server (server.R) - \emph{Her definere du det der skal til for at opbygge din løsning}
\item
  Kald til shinyApp function - \emph{Her bygges selev løsningen}
\end{itemize}

Du kan oprette en ny shiny App under:
\textbf{File -- New File -- Shiny Web App}
I dette vindue skal du indtaste navnet på din App og placeringen. Du har også mulighed for at vælge mellem og din App skal opdeles i to file, \textbf{ui.R} og \textbf{server.R}, eller om du vil have alt i en fil.

Opdelingen i to filer giver dig de fleste muligheder og en bedre kontrol.

Du får to filer der indeholder en demo løsning som du kan bruge som
udgangspunkt for din egen løsning.

\hypertarget{links}{%
\chapter{Links}\label{links}}

Her er en lille samling af nyttige links, det er på ingen måde en fuldstændig liste -- Brug Google.

\begin{itemize}
\tightlist
\item
  www.rproject.org
\item
  mran.microsoft.com/open
\item
  www.rstudio.com
\item
  support.rstudio.com/hc/en-us/categories/200035113-Documentation
\item
  www.statmethods.net
\end{itemize}

  \bibliography{book.bib,packages.bib}

\end{document}
